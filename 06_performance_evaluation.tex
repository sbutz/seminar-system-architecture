\section{Performance Evaluation}
\label{sec:performance_evaluation}

The performance evaluation of AdaptiveCpp’s stdpar backend is based on the benchmarking results provided in
Aksel Alpay and Vincent Heuveline’s paper \textit{AdaptiveCpp Stdpar: C++ Standard Parallelism Integrated Into
a SYCL Compiler}~\cite{alpay2021}. Due to the lack of a complete independent test setup, this section summarizes
the key findings from the original paper and discusses their implications. Readers interested in a detailed
description of the experimental methodology, hardware configurations, and benchmarking tools are referred to
the original work.

\subsection{Comparative Performance Analysis}
\label{sec:comparative_performance}

A key aspect of evaluating AdaptiveCpp’s stdpar backend is its performance relative to existing implementations.
The experimental results highlight the following comparisons:

\paragraph{Overall Speedup}
The execution of stdpar workloads using AdaptiveCpp’s SYCL backend achieves notable speedups when compared
to traditional CPU execution. Compared to NVIDIA’s proprietary \texttt{nvc++} stdpar backend, AdaptiveCpp
demonstrates competitive performance while maintaining broader hardware compatibility.

\paragraph{Kernel Launch Efficiency}
The reduction in kernel launch overhead, facilitated by AdaptiveCpp’s instant kernel submission model, results
in improved execution times. By minimizing redundant synchronization and event dependencies, kernel execution
latency is reduced, leading to a more efficient mapping of stdpar operations to SYCL kernels.

\paragraph{Memory Optimization Benefits}
The introduction of USM pooling in AdaptiveCpp significantly reduces memory allocation and transfer overhead.
By ensuring that memory is preallocated and reused efficiently, AdaptiveCpp minimizes host-device memory
transfer bottlenecks, particularly in workloads that involve frequent memory access.

\paragraph{Synchronization Efficiency}
The optimizations in event dependency management and lazy synchronization further contribute to reduced
execution stalls. The paper’s results confirm that the removal of redundant synchronization barriers improves
overall execution throughput without compromising correctness.

\subsection{Scalability Analysis}
\label{sec:scalability_analysis}

The scalability of AdaptiveCpp’s stdpar backend is evaluated across various hardware configurations and dataset
sizes. The findings indicate the following trends:

\paragraph{Hardware Scalability}
Performance improvements scale effectively across different GPU architectures and CPU-based parallel
execution environments. AdaptiveCpp enables seamless execution across heterogeneous devices, demonstrating
strong adaptability to varying compute resources.

\paragraph{Workload Scaling}
For large workloads, AdaptiveCpp’s optimizations ensure sustained performance improvements over traditional
CPU execution. However, for small workloads, the effectiveness of GPU offloading is more variable, reinforcing
the importance of the conditional offloading heuristic that keeps small computations on the CPU.

\subsection{Summary of Performance Findings}
\label{sec:performance_summary}

The experimental evaluation confirms that AdaptiveCpp’s stdpar backend provides an efficient mapping of
high-level parallel algorithms onto SYCL’s explicit execution model. The key takeaways from the performance
analysis include:

\begin{itemize}
    \item AdaptiveCpp achieves competitive execution performance compared to vendor-specific stdpar
          implementations while maintaining hardware portability.
    \item Kernel launch overhead is minimized through optimizations in task submission and event dependency
          tracking.
    \item Memory transfer overhead is significantly reduced due to USM pooling and optimized buffer
          management strategies.
    \item Synchronization optimizations effectively eliminate redundant barriers, leading to improved
          execution efficiency.
    \item Performance scaling across different hardware architectures validates AdaptiveCpp’s adaptability
          to various compute environments.
\end{itemize}

These findings demonstrate the effectiveness of AdaptiveCpp’s approach in bridging the gap between stdpar’s
high-level abstraction and SYCL’s explicit parallel execution model. Future work may further explore additional
compiler optimizations to enhance performance in specific workload scenarios.
