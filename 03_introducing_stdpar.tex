\section{Introduction to C++ Standard Parallelism (stdpar)}
\label{sec:stdpar}

The evolution of C++ has consistently aimed to enhance performance and efficiency. A significant milestone in
this journey was the introduction of standard parallelism, commonly referred to as \textbf{stdpar}, in the
C++17 standard. This feature provides developers with a standardized and portable approach to writing parallel
code, facilitating the efficient utilization of modern multi-core processors and accelerators~\cite{V'yukova2018}.

\subsection{Background and Motivation}

Prior to stdpar, parallel programming in C++ often relied on external libraries or platform-specific
extensions, such as OpenMP or Intel's Threading Building Blocks (TBB). While effective, these solutions lacked
standardization, leading to portability challenges across different compilers and platforms. The need for a
unified approach to parallelism in C++ became evident, prompting the inclusion of parallel execution policies
in the C++17 standard~\cite{V'yukova2018}.

\subsection{Overview of stdpar}

stdpar introduces execution policies that dictate how standard library algorithms are executed:

\begin{itemize}
  \item \textbf{\texttt{std::execution::seq}}: Specifies sequential execution.
  \item \textbf{\texttt{std::execution::par}}: Enables parallel execution across multiple threads.
  \item \textbf{\texttt{std::execution::par\_unseq}}: Allows parallel execution with vectorization, combining
  multithreading and SIMD (Single Instruction, Multiple Data) optimizations.
\end{itemize}

These policies can be applied to standard algorithms to control their execution behavior~\cite{V'yukova2018}.

\subsection{Example Usage of stdpar}

Here's a simple example demonstrating the use of \texttt{std::execution::par} to parallelize a standard
algorithm:

\begin{lstlisting}[language=C++, caption={Parallel execution using \texttt{std::execution::par}}]
#include <algorithm>
#include <execution>
#include <vector>

int main() {
    std::vector<int> data(100000, 1);
    std::for_each(std::execution::par, data.begin(), data.end(), [](int& x) { x *= 2; });
}
\end{lstlisting}

In this example, \texttt{std::for\_each} applies the provided lambda function to each element in the vector
\texttt{data} in parallel, effectively doubling each element's value.

\subsection{Execution Model, Hardware Mapping, and Memory Management}

The execution policy specified (e.g., \texttt{std::execution::par}) serves as a hint to the compiler and
runtime about how the algorithm should be executed. The actual execution can vary depending on the system's
hardware and the compiler's implementation. For instance, parallel execution may be achieved through
multi-threading on a CPU or offloaded to a GPU if supported. This abstraction allows developers to write
parallel code without delving into the specifics of the underlying hardware~\cite{V'yukova2018}.

Memory management in stdpar is typically handled by the compiler and runtime system. When parallel algorithms
are executed, the necessary data is usually allocated in host memory. If the computation is offloaded to an
accelerator, such as a GPU, the compiler and runtime manage data transfers between the host and the device.
For instance, NVIDIA's implementation of stdpar utilizes CUDA Unified Memory, allowing allocated memory to be
visible on both the host and the device, thereby simplifying memory management for the programmer. However,
depending on the implementation, explicit memory transfers may still be required to optimize performance for
specific workloads~\cite{NVIDIA2020}.

\subsection{Performance Considerations and Limitations}

While stdpar offers several advantages, such as portability and ease of use, there are considerations and
limitations to keep in mind:

\begin{itemize}
  \item \textbf{Backend-Dependent Performance}: The efficiency of parallel execution depends on the compiler's
  implementation and the target hardware. Not all compilers optimize stdpar for all architectures. For
  instance, NVIDIA's \texttt{nvc++} compiler provides limited support for C++ Parallel Algorithms on Pascal
  architecture GPUs, as they lack independent thread scheduling necessary to properly support the
  \texttt{std::execution::par} policy~\cite{NVIDIA2020}.
  \item \textbf{Lack of Fine-Grained Control}: Compared to low-level parallelism models like CUDA or OpenMP,
  stdpar provides less control over thread management and workload distribution. This abstraction can lead to
  suboptimal performance in scenarios requiring fine-tuned parallelism.
  \item \textbf{Compiler Support Variability}: Not all compilers fully support stdpar, and the level of
  optimization can vary between compiler implementations. For example, as of certain versions, Clang does not
  support C++17 execution policies~\cite{V'yukova2018}.
  \item \textbf{Language Feature Restrictions}: Certain C++ language features and constructs should be used
  with caution to ensure safe and efficient parallel execution. For example, the use of function pointers
  within parallel algorithms may be restricted or unsupported, depending on the compiler and target hardware.
  Additionally, exception handling within parallel regions is often limited, as many implementations do not
  support exceptions in GPU code~\cite{NVIDIA2020}.
\end{itemize}

The following sections will explore how SYCL can be leveraged as a backend for stdpar to address some of these
limitations and provide a more portable and efficient parallel programming model.
