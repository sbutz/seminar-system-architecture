\section{Motivation}
% 1 Seiten

% 1. Was ist das Thema? Warum ist das Thema für die Forschung relevant? Welche Forschungsfragen
% werden in der Arbeit beantwortet?
% Einleitung des Themas, Motivation

Modern hardware accelerators, such as GPUs and FPGAs, offer significant computational power. They are essential
for improving the performance of many applications. However, using these accelerators is challenging because they
require specialized knowledge of their architectures and programming models.

To address this, many libraries and frameworks have been created. Low-level libraries like CUDA for NVIDIA GPUs 
and ROCm for AMD GPUs provide detailed control over hardware. However, they also demand in-depth understanding of 
hardware-specific programming. This makes them difficult for developers who are not experts in high-performance 
computing.

Higher-level frameworks aim to simplify programming for accelerators while maintaining good performance. OpenCL, 
for instance, provides a platform-neutral model that works across CPUs, GPUs, and other accelerators. SYCL builds 
on OpenCL, offering a modern C++ interface that aligns with standard C++ practices. A key strength of SYCL is its 
multi-platform support, allowing developers to write portable code that can run on a variety of hardware, including 
GPUs, CPUs, and FPGAs, without modification. This feature ensures that applications can scale across diverse hardware 
architectures, providing flexibility and ease of use.
\cite{reinders2020data}

Despite these advancements, many frameworks still require developers to learn programming models and manage tasks 
like device selection and memory transfers. This adds complexity and slows adoption, especially for those outside 
specialized computing fields. Simplifying this process can save time and reduce the learning curve for developers.

Cost is another important factor. While accelerators improve computational performance, they also require substantial 
investments in hardware. Additionally, the time developers spend learning and coding for these systems contributes 
to project costs. Simplifying the programming process helps reduce these expenses. It allows developers to focus on 
solving domain-specific problems rather than hardware-specific challenges. Faster project completion also brings 
business benefits, such as quicker time-to-market and improved competitiveness.

C++ is one of the most widely used programming languages. It is known for its performance, flexibility, and large 
ecosystem. With the introduction of parallel algorithms in C++17, the language has made a major step towards 
simplifying parallel programming. These algorithms allow developers to express parallelism at a high level, avoiding 
the need to understand the details of the underlying hardware.

In their research, Aksel Alpay and Vincent Heuveline demonstrated how the SYCL programming model can act as a backend 
for C++ standard parallelism. A key advantage of this approach is that developers can use idiomatic C++ without having 
to learn a new programming model. By simply utilizing AdaptiveCpp’s compiler, developers can seamlessly integrate SYCL 
into their workflows. This approach ensures that parallelism is implemented in a familiar C++ environment while still 
achieving portability and efficiency across various hardware platforms. This combination of SYCL and C++ standard 
parallelism creates a unified and accessible approach to modern parallel programming.
\cite{alpay2021}

As the foundation of this work, understanding the SYCL programming model is essential. The next section introduces 
SYCL, its core principles, and its design, explaining how it provides a modern, high-level interface for heterogeneous 
computing while maintaining compatibility with standard C++ practices.
