\begin{abstract}
Modern hardware accelerators such as GPUs and FPGAs provide immense computational power, yet their
utilization remains challenging due to the complexity of existing programming models. While low-level
solutions such as CUDA and OpenCL offer high performance, they require specialized knowledge,
limiting accessibility. High-level approaches like SYCL aim to bridge this gap by providing a modern,
C++-based abstraction for heterogeneous computing.

This paper explores the integration of SYCL as a backend for C++ Standard Parallelism (stdpar) through
the AdaptiveCpp compiler. By leveraging compiler-driven transformations, this approach enables efficient
execution of stdpar workloads on heterogeneous devices while maintaining the high-level abstraction
expected from the C++ standard library. Key challenges in execution, memory management, and
synchronization are identified, and solutions leveraging SYCL’s task-based execution model and Unified
Shared Memory (USM) are presented.

Performance optimizations, including kernel launch overhead reduction, memory pooling, and conditional
offloading, are analyzed to demonstrate the effectiveness of this integration. The evaluation confirms that
the AdaptiveCpp approach provides competitive performance compared to vendor-specific implementations
while maintaining portability across different hardware architectures.

This study highlights the broader implications of compiler-assisted parallelism, demonstrating how tighter
integration between C++ standard features and heterogeneous programming models can simplify adoption
and improve efficiency. Future C++ standards, including senders/receivers and std::simd, will further
enhance the expressiveness and performance of parallel programming in modern software development.
\end{abstract}
